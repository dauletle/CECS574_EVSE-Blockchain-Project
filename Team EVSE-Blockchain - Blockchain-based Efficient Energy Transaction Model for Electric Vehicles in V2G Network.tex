% Options for packages loaded elsewhere
\PassOptionsToPackage{unicode}{hyperref}
\PassOptionsToPackage{hyphens}{url}
%
\documentclass[
]{article}
\usepackage{lmodern}
\usepackage{amssymb,amsmath}
\usepackage{ifxetex,ifluatex}
\ifnum 0\ifxetex 1\fi\ifluatex 1\fi=0 % if pdftex
  \usepackage[T1]{fontenc}
  \usepackage[utf8]{inputenc}
  \usepackage{textcomp} % provide euro and other symbols
\else % if luatex or xetex
  \usepackage{unicode-math}
  \defaultfontfeatures{Scale=MatchLowercase}
  \defaultfontfeatures[\rmfamily]{Ligatures=TeX,Scale=1}
\fi
% Use upquote if available, for straight quotes in verbatim environments
\IfFileExists{upquote.sty}{\usepackage{upquote}}{}
\IfFileExists{microtype.sty}{% use microtype if available
  \usepackage[]{microtype}
  \UseMicrotypeSet[protrusion]{basicmath} % disable protrusion for tt fonts
}{}
\makeatletter
\@ifundefined{KOMAClassName}{% if non-KOMA class
  \IfFileExists{parskip.sty}{%
    \usepackage{parskip}
  }{% else
    \setlength{\parindent}{0pt}
    \setlength{\parskip}{6pt plus 2pt minus 1pt}}
}{% if KOMA class
  \KOMAoptions{parskip=half}}
\makeatother
\usepackage{xcolor}
\IfFileExists{xurl.sty}{\usepackage{xurl}}{} % add URL line breaks if available
\IfFileExists{bookmark.sty}{\usepackage{bookmark}}{\usepackage{hyperref}}
\hypersetup{
  pdftitle={Blockchain-based Efficient Energy Transaction Model for Electric Vehicles in V2G Network},
  hidelinks,
  pdfcreator={LaTeX via pandoc}}
\urlstyle{same} % disable monospaced font for URLs
\usepackage{graphicx}
\makeatletter
\def\maxwidth{\ifdim\Gin@nat@width>\linewidth\linewidth\else\Gin@nat@width\fi}
\def\maxheight{\ifdim\Gin@nat@height>\textheight\textheight\else\Gin@nat@height\fi}
\makeatother
% Scale images if necessary, so that they will not overflow the page
% margins by default, and it is still possible to overwrite the defaults
% using explicit options in \includegraphics[width, height, ...]{}
\setkeys{Gin}{width=\maxwidth,height=\maxheight,keepaspectratio}
% Set default figure placement to htbp
\makeatletter
\def\fps@figure{htbp}
\makeatother
\setlength{\emergencystretch}{3em} % prevent overfull lines
\providecommand{\tightlist}{%
  \setlength{\itemsep}{0pt}\setlength{\parskip}{0pt}}
\setcounter{secnumdepth}{-\maxdimen} % remove section numbering

\title{\protect\hypertarget{_rock487eds43}{}{}Blockchain-based Efficient
Energy Transaction Model for Electric Vehicles in V2G Network}
\author{}
\date{}

\begin{document}
\maketitle

Diego Aulet-Leon

\emph{Graduate Student}

California State University, Long Beach

Long Beach, California, USA

\href{mailto:dauletle@gmail.com}{\nolinkurl{dauletle@gmail.com}}

Vatsal Patel

\emph{Graduate Student}

California State University, Long Beach

Long Beach, California, USA

\href{mailto:Vatsal.Patel01@student.csulb.edu}{\nolinkurl{Vatsal.Patel01@student.csulb.edu}}

\textbf{Abstract - Vehicle to Grid (V2G) technology allows for electric
vehicles (EVs) to supply power back to the grid via EV charging
stations. The concept of V2G has a wide variety of uses, from storing
power generated by renewables to be used during higher-demand hours, to
becoming an essential source of power during environmental disasters.
While the technology has been widely accepted, the implementation has
been limited due to the restrictions the power utilities have in buying
and selling power, particularly, the restrictions imposed with all of
the banking systems the currency has to flow through to complete a
transaction. Blockchain-based energy provides a solution of solving this
problem by decentralizing the energy transaction away from the power
utilities and third parties, and thus, make V2G a reality. This paper
will go over the different types of Blockchain strategies that can be
implemented to achieve the goal of decentralized energy transactions for
V2G functionality.}

\textbf{Keywords---Electric vehicles, charging, blockchain, energy
transaction, vehicle-to-grid (V2G), multi-agent coalition, edge as a
service.}

I.~ INTRODUCTION

In the last decade, the biggest push towards curbing climate change has
been the push to widely used renewable energy sources, such as
photovoltaics (solar), and wind power. While the supply of power has
grown to make renewables a cheaper source of power than power generated
by fossil fuels {[}7{]}, the biggest setback has been the inflexibility
of using power generated from renewables during times the renewable
power cannot be generated. Specifically, power generated by
photovoltaics can only be generated at the middle of the day, and wind
power can only be generated when there is a sufficient amount of wind.
As a result, the demand for power through a 24-hour period has shifted
so that there is a much lower demand of power at the middle of the day,
which can be represented in the ``duck curve'' shown below.

\includegraphics[width=3in,height=1.95833in]{media/image1.png}

Figure 1: "Duck Curve", daily power grid demand. {[}3{]}

The issue with this fluctuation in demand for power is that a
significant amount of power needs to be generated when the demand for
power dramatically increases. As of now, the best way to provide this
immediate demand is to use dirtier power sources, such as coal or
natural gas, where the supply of power can quickly be generated by
supplying more fuel in the power station. This, as a result, creates a
dependency for power generated by fossil fuels, which worsens climate
change.

There have been many suggested ideas to help remove this dependency, and
``flatten the power demand curve''. The most logical idea is to have a
large enough battery to store the power generated by renewables during
their peak hours, so that this power can be used during the peak demand
later in the day. While there is work being done to create a large
enough ``battery'', there currently isn't anything available to store
the thousands of megawatts needed to respond to the quick demand.

Meanwhile, another approach to curve climate change has been the growing
adoption of Electric vehicles (EVs). EVs have been gaining remarkable
popularity throughout the world over the past decade, not only because
they are more cost effective for the driver, but also for their ability
to produce lower emissions by the efficient utilization of renewable
energy. However, one of the most interesting promises is the advent of
using the energy already stored in electric vehicles to power the grid
during peak hours. This process of asserting power back to the grid is
known as vehicle-to-grid (V2G). With this idea, the idle EVs can deliver
their unused electrical energy stored in their batteries to the power
grid for money; thus, create an opportunity of electricity trading in
vehicular networks. Moreover, a massive deployment of V2G technology can
be combined as the ``battery'' needed to handle the instant demand for
power that occurs in the evening. This would not only reduce costs by
reducing the necessity to establish additional power plants to handle
the instant demand, but also remove the need for dirtier power sources,
and continue the goal to reduce carbon emissions.

The idea and promise of V2G has been around for over a decade. However,
the main reason why it isn't widely adopted is due to how complicated it
would be to apply this system into our current power grid. For example,
if one were to attempt to supply power from their electric vehicle back
to the power grid, the utility companies involved with generating the
power (from the power station, to the power transmission companies, to
the local electrical utility) would all have to ``buy'' the power from
the owner of the EV. In the current model, the business model for the
utility companies are structured for power to only go in one direction,
where restructuring this model to receive power from the customer is
unfeasible and unsustainable. In addition, having the power distribution
reliant on the power grid creates a security risk, where a malicious
hacker would have a single point of attack, the power utility.
Therefore, a new distribution model would need to be created to (1)
determine the electricity trading price and amount, and (2) settle the
trading price contract in an effective and secure manner.

With the model stated, the most reasonable solution to overcome this
problem is with using Blockchain. Blockchain was built as a distributed
ledger that permits transactions to be gathered into blocks and
recorded. It allows the resulting ledger to be accessed by different
servers and which cryptographically chains blocks in chronological order
{[}6{]}. By setting up each customer as a node in the Blockchain system,
one can achieve the ability of utilizing V2G transactions, and thus,
improving the ability to curb climate change.

This paper has been organized to go over two approaches discussed for
using Blockchain in the power grid. Following the analysis, a conclusive
opinion would be discussed, as well as future actions to take for
implementation.

II.~ \textsc{Multi-Agent Coalition {[}2{]}}

In the paper provided by Fengji Luo and his team in IEEE, one form
proposed of using Blockchain for EV Charging is to use a multi-agent
coalition to perform the negotiations for the energy transactions that
would happen between prosumers (producers and consumers). According to
the paper, ``Multi-agent coalition refers to a way to cooperate agents
to complete a task, where none of them can complete it independently.''
In the first part of the model, each prosumer is modelled as a
Multi-Agent System (MAS), which is set as the lowest layer in the
system. Each MAS has multiple Resource Agents (RAs) in the layer above,
where the RAs are meant to perceive the operational status of energy
resources from the MASs, and send the information to the Local
Coordination Agent (LCA) in the next layer. Based on the energy resource
information from RAs, the LCA automatically performs the energy
management by ``solving a local scheduling model to determine optimal
control actions of the internal energy resources with the aim to serve
local power consumption''. If the LCA determines that the local
generation cannot serve the consumption of the MASs, it is deemed as an
energy shortage. In this case, the Social Coordination Agent (SCA) for
the LCA negotiates with the grid and other corresponding SCAs to
purchase the power at the lowest cost. The SCA is located at the highest
layer, and can offer to sell excess power based on the amount of excess
power it has. The organization of this multi agent coalition is shown
below.

\includegraphics[width=3in,height=3.23611in]{media/image2.png}

Figure 2: Multi Agent Coalition Model

The second part of this model uses Blockchain to settle the energy
transactions. The Blockchain essentially consists of a parallel
double-chain; one chain is for the contract and the other chain is for
the ledger. Each block in the contract chain contains only one energy
trading contract that is placed in sequential order. The ledger chain is
the final financial settlement which corresponds to each contract block.
Each pair of contract-block and ledger-block is equipped with a high
frequency verifier that works uninterrupted once generated.

\includegraphics[width=3in,height=2.125in]{media/image3.png}

Figure 3: Blockchain double-chain model.

The energy negotiation process is handled with the multi agent
coalition, where each of them are dependent on one another. The diagram
below provides an overview of how the negotiation is processed.

\includegraphics[width=3in,height=3.65278in]{media/image4.png}

Figure 4: Multi agent energy negotiation state diagram.

As seen in the diagram, the negotiation workflow begins with the
prosumer continuously monitoring its on-site local energy resources and
performs autonomous energy management. This repeats until there is an
energy shortage, where once there is a shortage, the prosumer acts as a
buyer to launch electricity negotiation requests. This describes the
first formulation, the Autonomous Energy Management of Prosumer. The
goal of this model is that the LCA uses the electricity retail price
signed by the prosumer to calculate the price of power which the
prosumer needs to buy from external suppliers.

For each prosumer, the RAs monitor the energy resources through sensors,
and store data into local storage. Based on the historically recorded
data, the LCA performs very short-term forecasting to predict the power
generation and consumption over future time intervals. Once a
forecasting result is determined, the LCA solves a local scheduling
model to allocate power outputs of the on-site generation resources to
serve the local load, while satisfying operational constraints. With
this algorithm, the LCA determines control schedules of local producers
and consumers, and the amount of intended energy purchase. Then, if the
prosumer does need to buy additional power, the LCA forwards the deficit
energy to the SCA, and the latter will launch the coalition request to
try to buy energy from other prosumers with a price lower than the
electricity retail price signed by the prosumer. Once this is done, the
Agent Coalition Formulation algorithm is executed to determine the
contract to use for prosuming energy.

A.~ \textsc{Agent Coalition Algorithm}

The algorithm begins with the electricity purchase information sent from
LCA, where it is denoted in an array of blocks of intended electricity
purchases. Each block includes tuples representing the start time, end
time, and energy amount (kWh) of the electricity purchase. By receiving
the blocks of purchases from the LCA, the SCA starts a coalition request
for each block. Firstly, the SCA of prosumer (the buyer) initializes its
neighbored SCA list. Then, the buyer contacts each adjacent SCAs to
negotiate electricity trading before the deadline. In this process, the
buyer calculates its available energy based on perceiving its connection
line's power capacity. The buyer also propagates the request to
neighbors of a randomly selected SCA if all following three conditions
are satisfied: (1) the trading deadline is not reached; (2) if there is
at least one SCA not in the adjacent SCA list; and (3) the request
propagation depth is less than the pre-specified threshold. Here the
control parameter threshold controls the times the buyer can propagate
the request through its connected SCAs.

After the trading deadline arrives, the buyer begins the price
negotiation using the next algorithm, the Electricity Trading
Negotiation Protocol.

B.~ \textsc{Electricity Trading Negotiation Algorithm}

The algorithm starts with the seller receiving the buyer's request. If
the seller has available electricity during the period of the task, the
buyer sends a reply to the seller and starts the electricity trading
negotiation process. The proposed negotiation protocol of electricity
trading is based on the alternating offers protocol. Firstly, the seller
provides an offer to the buyer. The energy selling price in the offer is
based on the evaluation of the seller's generation cost and the number
of existing temporary contracts the seller has, which is dependent on
the generation cost of serving the electricity demand of the task. The
value of the generation cost is determined by the LCA of the seller
through local scheduling.

\includegraphics[width=3in,height=1.33333in]{media/image5.png}

Figure 5: Trading negotiation algorithm diagram.

By receiving the offer, the buyer has three options: accept the offer,
reject the offer, or generate a counter-offer to the seller. The
rationale of the model is if the total contracted energy amount of the
temporary contracts owned by the seller is large enough and the
counteroffer price is smaller than the average price of temporary
contracts, then the seller will reject the counteroffer; otherwise, the
counteroffer will be accepted. The rationale of the model is that when
generating an offer, the seller raises the selling price with the
increase of the number of its existing temporary contracts. Also, the
model shows that when the buyer has more temporary contracts in hand,
its intended purchase price decreases.

When the trading deadline arrives, the negotiation is finished, and the
Final Contract Determination Algorithm can begin.

C.~ \textsc{Final Contact Determination Algorithm}

Final contracts are selected from the temporary contract set in this
algorithm, which is populated with the negotiation tasks from the
previous algorithm. The final contract determination process is launched
by the buyer. Firstly, if the contract determination deadline does not
arrive, the buyer stacks its owned temporary contracts, and selects
temporary contracts with the price from high to low. Once a contract is
selected, the buyer sends a transaction request message to the seller.
When the seller receives the transaction request message, it firstly
checks whether it has adequate energy capacity to accomplish this
contract excluding existing final contracts. If so, then the seller
sends a confirmation message to the buyer, and the contract will be
finally confirmed as a final contract. Otherwise, the seller sends a
cancel message to the buyer, the temporary contract will be cancelled,
and the buyer proceeds to the next contract in its stacked contract
list. This occurs until one of following three conditions is satisfied:
(a) the sum of final contracted capacity is reached; (b) all the
temporary contracts have been processed; and (c) the final contract
determination deadline arrives. When the final contract determination
deadline arrives, all the remaining temporary contracts will be
cancelled.

D.~ \textsc{Blockchain Based Transaction Mechanism}

All prosumers in the Blockchain based energy trading community compose a
private distributed network, in which only registered prosumers can
participate in the energy trading processes. The contract-chain starts
to get verified by the system when the buyer broadcasts the final
contract to all other prosumers (nodes). Once received, other nodes
decrypt the contract to confirm whether the contract is agreed by both
the buyer and the seller. The verification result is then reviewed by
the voting of all nodes, in which each node has precisely one chance to
vote. Only when the majority of nodes agree, the contract is considered
as valid and written in a new contract block. Lastly, each prosumer then
generates a hash digest by using SHAs to chain the new contract block to
the existing chain. Once a new contract block is chained onto the
contract chain, each node calculates the ledger to update the balance
after the execution of the new contract. A randomly selected node is
responsible for broadcasting its calculated ledger to all other
prosumers so as to let all nodes make a contradistinction between their
calculated ledgers and the received one. The new ledger is considered
through a prosumer motivated voting mechanism as valid, then written in
a new ledger block. When a ledger block is chained onto the existing
ledger chain, a high frequency verifier, which is located on each node,
is triggered immediately to work uninterruptedly for the new generated
contract-block and ledger-block pair.

E.~ \textsc{Simulation of a Multi Agent Coalition}

In the simulation done in the paper, the load curve of the prosumers
were generated based on the Australian ``Smart Grid, Smart City''
dataset, in which the electricity consumption data of more than 300
Australian residential users in the state of New South Wales were
recorded. In the paper's simulation discussion, each prosumer had
simulated rooftop solar or wind power sources. While the on-site battery
energy storage system (BESS) is simulated as a power cell installed in a
residential home, the BESS can also be assumed to be a vehicle's battery
system, which can store from 15-100kWh of power. A twenty four hour
horizon was simulated, with the duration of each time interval set to be
30 minutes, and the negotiation deadline and final contract
determination deadline were set. Below are the result found in the
simulations:

\begin{itemize}
\item
  The number of tasks increased in an approximately linear relationship,
  which is mainly because in this simulation 50 prosumers were set as
  base configurations, and other prosumers were replicated with some
  variations.
\item
  Assuming that communication network traffic is not considered, the
  average negotiation time was less than one second (about 700 ms) when
  there were 300 prosumers. While this proves the efficiency of the
  proposed system, the reality is that the negotiation time would depend
  on the real-time network traffic conditions.
\item
  In the simulation, about 75-80\% of deficit electricity transactions
  were satisfied by trading electricity with other prosumers. Only for a
  small proportion of electricity did the prosumers have to purchase the
  energy from the grid at the retail price.
\item
  By primarily purchasing power from other prosumers, this not only
  ideally saves costs for the prosumers (by purchasing power from other
  prosumers at a lower cost), but also allows for a reduced demand of
  power from the power grid, which acquires its power from larger power
  stations.
\end{itemize}

\includegraphics[width=3in,height=1.93056in]{media/image6.png}

Figure 6: Simulation of power demand in New South Whales (NSW), adjusted
with implemented multi agent coalition.

The figure above illustrates the 24-hour internal energy resource
scheduling performed by the LCA of a single prosumer. The red line
represents the normal load curve of the prosumer, or in other words, the
power demanded by the prosumer. The graph shows the stacked areas form
the shifted load curve produced by the LCA, which are served by
different energy sources. From this graph, it is evident that most of
the power is either purchased from other prosumers, or acquired from the
battery storage of the battery system. However, what is most significant
is when the power is purchased from the grid, where the power is
purchased when there is a peak supply of power from photovoltaic power.
The reason for its significance is that it highlights how the model can
shift the power dependency to renewable energy, while allowing for the
demands of power that occur throughout the day to be satisfied.

\hypertarget{iii.-edge-as-a-service-5}{%
\subsubsection{\texorpdfstring{III.~ \textsc{Edge as a Service
{[}5{]}}}{III.~ Edge as a Service {[}5{]}}}\label{iii.-edge-as-a-service-5}}

In the paper provided by Xuesong Xu and his team in Sensors (Basel), a
Layered Lightweight Blockchain Framework (LLBF) was designed. The nodes
in the LLBF are composed of various edge computing devices, including
execution units, centralized controllers, network devices and servers.
In order to ensure the scalability of blockchains and reduce network
delay, edge devices or edge computing units in resource constrained
layers (RCL) are grouped by functional attribute clustering, and each
cluster selects cluster head (CH) to manage the corresponding
blockchain. If there is excessive delay of some nodes in the industrial
internet, then these nodes can be re-clustered and change their
clusters. Generally, CH nodes that remain online for a long time are
selected and the basic tasks in the cluster are stable. Therefore, the
RCL blockchain is not affected by the dynamic changes of devices.
Asymmetric encryption, digital signature and cryptographic hash
functions, such as SHA256, are utilized to protect all kinds of
transactions generated by nodes. A transaction structure of block data
in LLBF is shown in Figure below, which contains seven attribute fields
and one data field. The first field records the current transaction ID,
and the second field is the aforementioned transaction pointer, which is
linked into blocks through pointers. The next four fields are the PK and
digital signature. The digital signature belongs to the requester and
requestee. The seventh field is the Output {[}i{]}, i = 0,1,2 set of the
requester. Output {[}0{]} denotes the number of accepted transactions
generated by requester, Output {[}1{]} denotes the number of
transactions rejected by requested, and Output {[}2{]} is the PK hash
used in the next transaction of requester. The final field is
``metadata'', which provides a record of the operations required by the
device node, including ID, device name, and operation type.

\includegraphics[width=3in,height=1.15278in]{media/image7.png}

Figure 7.: Cluster Head (CH) diagram

Each CH decides independently whether to retain or discard a new block
based on the communication received from the transaction participant
(including the requester and the requested), which may result in
different versions of blocks in each CH. Corresponding to the accounting
operation of each cluster center node, the blockchain model of this
layer does not need to coordinate the block consistency in real time,
thus reducing the block synchronization overhead. Data resources are
summarized through the resource extended layer (REL) layer, so the
resource extended nodes in the REL verify the block consistency within a
specified waiting period, thus addressing the problem of insufficient
computing performance of equipment resources in RCL layer.

\includegraphics[width=3in,height=1.90278in]{media/image8.png}

Figure 8: Network structure model of Layered-blockchain

The above figure describes the network structure model of
layered-blockchain, and Figure below presents the design of blocks of
RCL layer.

Each RCL block contains a block head and a policy head. The block-head
stores the previous block hash, and the policy head maintains an access
control list. These control lists define RCL transactions and rules for
communicating with REL. The policy head has four parameters. The first
parameter is the network device ID of the requester transaction. The
second parameter represents the request requirement, including data
writing, data reading, access control, monitoring and data transmission.
The third parameter is the specified target device, and the fourth
parameter is the operation permission. All block structures are
indicated on the left side of the figure, identifying the specific
content of a transaction, including the current transaction, transaction
chain, transaction type, access device and related operation record
information. Each REL block contains a block head and a transaction
body. The block-head stores the previous block hash value, generator ID,
and verifier's signature. If illegitimate access attempts to change
previously transactions, the hash value of the corresponding block will
remain on the block and there will be inconsistencies, thus exposing
this attack. In the RCL layer, because of the equipment resources
constraints, the CH block head only stores hash value, the block body
only stores basic control and state information. The initialization
maximum block body size is 256 Kb, and the total cache size of each
device block is 2 M. If the block continues to increase, the block will
be synchronized to the REL layer with greater processing capacity.

\includegraphics[width=5.94271in,height=2.87184in]{media/image9.png}

A.~ \textsc{Testing for Efficiency of the Network}

The figure below shows the impact of the number of validators and
invalid transactions in a block on the probability that no invalid
transactions are detected during block validation. As the number of
validators increased, we could reduce the percentage of transactions
that needed to be validated without compromising the performance of
invalid transaction detection. When each validator only validated 40\%
of the transactions, the probability of not detecting an invalid
transaction in 80 transactions was less than 0.3\%. If there were
multiple invalid transactions (when the number of invalid transactions
is 5, 10 respectively), the probability of not detecting invalid
transactions was significantly reduced.

Which makes this proposed method more secure and steady for the real
world implementation for EVs to plug into the V2G grid to stabilize the
need for energy generation from fossil fuel, hereby, making renewable
energy more reliable.

\includegraphics[width=2.875in,height=2.22569in]{media/image10.png}

\includegraphics[width=2.9081in,height=2.19792in]{media/image11.png}

Figure 9: The impact of the number of validators and invalid
transactions in a block.

\hypertarget{iv.-conclusions}{%
\subsubsection{\texorpdfstring{IV.~
\textsc{Conclusions}}{IV.~ Conclusions}}\label{iv.-conclusions}}

In this paper, we have proposed strategies to implement a
blockchain-based energy trading model for EVs in V2G network. P2P energy
transactions between charging stations and EVs without need of any
trusted third party are enabled by means of blockchain technology. The
transaction data is stored in the blockchain network in a distributed
manner using a multi-agent Coalition and edge as a service. Smart
monitoring of the energy usage and the remaining amount of energy of EVs
is enabled by installing smart meters into EVs and connecting the EVs to
the smart grid via the Internet. This method of using EVs extra
electricity for curbing up the need of generating electricity using
fossil fuel.

\begin{enumerate}
\def\labelenumi{\arabic{enumi}.}
\item
  \begin{quote}
  M. M. Islam, M. Shahjalal, M. K. Hasan and Y. M. Jang,
  "Blockchain-based Energy Transaction Model for Electric Vehicles in
  V2G Network," \emph{2020 International Conference on Artificial
  Intelligence in Information and Communication (ICAIIC)}, Fukuoka,
  Japan, 2020, pp. 628-630, doi: 10.1109/ICAIIC48513.2020.9065221.
  \end{quote}
\item
  \begin{quote}
  F. Luo, Z. Y. Dong, J. Murata, Z. Xu, and G. Liang, ``A distributed
  electricity trading system in active distribution networks based on
  multiagent coalition and blockchain,'' IEEE Trans. Power Syst., vol.
  34, no. 5, pp. 4097--4108, Sep. 2018.
  \end{quote}
\item
  \begin{quote}
  Jones-Albertus, Becca. ``Confronting the Duck Curve: How to Address
  Over-Generation of Solar Energy.'' Energy.gov, Office of Energy
  Efficiency and Renewable Energy, 12 Oct. 2017,
  www.energy.gov/eere/articles/confronting-duck-curve-how-address-over-generation-solar-energy.
  \end{quote}
\item
  \begin{quote}
  Jh. Liu, Y. Zhang, S. Zheng, and Y. Li, ``SURVIVOR: A blockchain based
  edge-as-a-service framework for secure energy trading in SDNenabled
  vehicle-to-grid environment,'' Comput. Netw., vol. 153, pp. 36--48,
  Apr. 2019
  \end{quote}
\item
  \begin{quote}
  Luo, Chuanwen \& Xu, Liya \& Li, Deying \& Wu, Weili. (2020). Edge
  Computing Integrated with Blockchain Technologies.
  10.1007/978-3-030-41672-0\_17.
  \end{quote}
\item
  \begin{quote}
  Sheedy, Chris. ``Blockchain: the Future of Record Keeping.''
  INTHEBLACK, 12 June 2019,
  www.intheblack.com/articles/2018/03/22/blockchain-future-record-keeping.
  \end{quote}
\item
  \begin{quote}
  Ellsmoor, James. ``Renewable Energy Is Now The Cheapest Option - Even
  Without Subsidies.'' Forbes, Forbes Magazine, 15 June 2019,
  www.forbes.com/sites/jamesellsmoor/2019/06/15/renewable-energy-is-now-the-cheapest-option-even-without-subsidies/?sh=103033b65a6b.
  \end{quote}
\end{enumerate}

\end{document}
